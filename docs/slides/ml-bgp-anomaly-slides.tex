
\documentclass{beamer}

% Configuración del tema de Beamer
\usetheme{Madrid}

% Paquetes necesarios
\usepackage[utf8]{inputenc}
\usepackage{graphicx}
\usepackage{hyperref}

% Información de la presentación
\title{Detección de Anomalías en BGP usando Machine Learning}
\author{Carlos Marcelo Martinez}
\date{Workshop IA en Sistemas Ciber-Físicos \\ Nov 2024 - Feb 2025}

\begin{document}

% Diapositiva de título
\begin{frame}
    \titlepage
\end{frame}

% Introducción
\begin{frame}{Introducción}
    \begin{itemize}
        \item El protocolo BGP es crítico para la conectividad global de Internet.
        \item Es vulnerable a incidentes como secuestro de rutas, filtraciones (leaks) y errores de configuración.
        \item Este trabajo explora el uso de Machine Learning para detectar anomalías en BGP.
    \end{itemize}
\end{frame}

% Objetivos del estudio
\begin{frame}{Objetivos}
    \begin{itemize}
        \item Identificar incidentes históricos relevantes de BGP.
        \item Extraer datos de BGP, identificando features útiles
        \item Aplicar técnicas de aprendizaje automático para detectar anomalías.
    \end{itemize}
\end{frame}


% Casos de Estudio
\begin{frame}{Casos de Estudio}
    \begin{itemize}
        \item \textbf{Apagón de Moscú 2005}: Impacto en el tráfico de Internet.
        \item \textbf{Filtración de rutas de Level 3 (2017)}: Afectación masiva en EE.UU.
        \item \textbf{Incidente de Rostelecom (2020)}: Filtración de miles de prefijos europeos.
    \end{itemize}
\end{frame}


% Fuentes de Datos
\begin{frame}{Fuentes de Datos}
    \begin{itemize}
        \item Datos obtenidos de:
        \begin{itemize}
            \item \textbf{RIPE RIS}: Conjunto de datos de enrutamiento global, generado y mantenido por el RIPE NCC (Netowrk Coordination Centre).
        \end{itemize}
        \item Procesamiento usando \textbf{BGPStream} para extraer features.
    \end{itemize}
\end{frame}

% Features
\begin{frame}{Features}
	\begin{itemize}
		\item 0: update\_count - The number of various BGP messages received over time
		\item 1: announcement\_count | The number of BGP announcements over time
		\item 2: withdrawals\_count | The number of BGP withdrawals over time
		\item 3: as\_path\_len\_avg | The average length of the AS Path contained in BGP updates
		\item 4: as\_path\_len\_median | The median / average length of the AS Path contained in BGP updates
		\item 5 : as\_path\_change | The number of prefixes that change the as-path in their announcements
	\end{itemize}	
\end{frame}


% Algoritmos Utilizados
\begin{frame}{Pruebas realizadas con diferentes algoritmos}
    \begin{itemize}
        \item \textbf{Supervisado}: 
        \begin{itemize}
            \item Regresión Logística.
            \item Árboles de Decisión y Random Forest.
        \end{itemize}
        \item \textbf{No Supervisado}:
        \begin{itemize}
            \item One-Class SVM.
            \item Local Outlier Factor (LOF).
        \end{itemize}
    \end{itemize}
\end{frame}

% One-Class SVM
\begin{frame}{One-Class SVM}
    \begin{itemize}
        \item \textbf{¿Qué es?} 
        \begin{itemize}
            \item Algoritmo de aprendizaje no supervisado basado en máquinas de soporte vectorial.
            \item Modela los datos normales y detecta outliers como desviaciones significativas.
        \end{itemize}
        \item \textbf{Resultados obtenidos:}
        \begin{itemize}
            \item Identificó correctamente intervalos anómalos en eventos conocidos.
            \item Ajuste de umbrales fue necesario para mejorar la precisión.
            \item Detectó anomalías en eventos como el apagón de Moscú 2005 y la filtración de Level 3.
        \end{itemize}
    \end{itemize}
\end{frame}



% Local Outlier Factor (LOF)
\begin{frame}{Local Outlier Factor (LOF)}
    \begin{itemize}
        \item \textbf{¿Qué es?}
        \begin{itemize}
            \item Algoritmo basado en la densidad de vecinos cercanos.
            \item Mide cuán aislado está un punto respecto a su vecindario.
        \end{itemize}
        \item \textbf{Resultados obtenidos:}
        \begin{itemize}
            \item Detectó con precisión eventos anómalos sin necesidad de datos etiquetados.
            \item Menos sensible a outliers extremos que One-Class SVM.
            \item Se observó un buen rendimiento en la detección de filtraciones de prefijos.
        \end{itemize}
    \end{itemize}
\end{frame}

% Resultados
\begin{frame}{Resultados}
    \begin{itemize}
        \item Los modelos supervisados sufrieron de **desbalanceo de clases** (la clase "normal" está sobre-representada) y bajo \textbf{recall}.
        \item Por el tipo de datos y features me resultó muy dificil etiquetar apropiadamente los datasets para aplicar métodos supervisados
        \item Los métodos no supervisados (One-Class SVM y LOF) lograron detectar intervalos anómalos con mayor precisión.
        \item Se identificó que las ventanas de tiempo cortas (\textbf{6 segundos}) mejoraban la detección de anomalías.
    \end{itemize}
\end{frame}


% Conclusiones
\begin{frame}{Conclusiones}
    \begin{itemize}
        \item Creo que logré resultados bastante prometedores :-).
        \item Los modelos no supervisados son más efectivos, pero requieren ajuste fino.
        \item Es necesario mejorar la precisión de los datos y explorar estrategias de detección en tiempo real.
    \end{itemize}
\end{frame}

% Trabajo Futuro
\begin{frame}{Trabajo Futuro}
    \begin{itemize}
        \item Implementación en tiempo real con \textbf{Kafka} y \textbf{procesamiento de streaming}.
        \item Mejora en la selección de features para detectar anomalías con menor cantidad de prefijos afectados.
        \item Desarrollo de modelos explicables que diferencien entre ataques, fallas de red y cambios benignos.
    \end{itemize}
\end{frame}

% Diapositiva de agradecimiento
\begin{frame}{¡Gracias!}
    \centering
    ¿Preguntas? \\
    \vspace{1cm}
    \textbf{Contacto:} \\
    \href{mailto:carlos@cagnazzo.uy}{carlos@cagnazzo.uy} \\
    \href{mailto:carlos@lacnic.net}{carlos@lacnic.net}
\end{frame}

\end{document}


